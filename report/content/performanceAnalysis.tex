\section{Agent Performance Analysis}\label{sec:performanceAnalysis}

Having presented MAS backend traffic simulation and frontend visualization procedures in \autoref{sec:backend} and \autoref{sec:frontend} respectively, we now add evaluation procedure and its implementation as well as resulting performance metrics.


\subsection{Evaluation Procedure}

There are different types of agents (of \textit{local} or \textit{global} nature) with randomly engaged origin and destination nodes as well as random events influencing the state of the network. Agents move along edges and thus reduce the speed of other agents proceeding to traverse said edge.

With ouputs of the underlying simulation on carAgents-, planner- and events-log together with the graph-file, all performance metrics can be retrieved from these information:
\begin{itemize}
    \item The planner-log allows to compute \textbf{expected travelling time} from the initiated path for an agent with the weights of the edges as depicted in the graph-file
    \item From 'spawn'- and 'reach'-events in the carAgents-log, the \textbf{actual travelling time} can be computed. Note that the actual travelling time may never be larger than the expected travelling time, as events in the network never speed-up edge traversal and traversal time gets impacted only negatively by other car-agents.
    \item Subtracting expected from actual travelling time yields the agent's \textbf{travelling time discrapency}, which depicts one of the major performance number
    \item Also note that the carAgent-log's 'spawn'-event holds information regarding the \textbf{type of car-agent} spawned. Travelling time information is persisted and compared with regards to the respective car-agent type as well.
    \item The number of reroutes for car-agents of type 'global' shall also be considered. 
\end{itemize}

The above depicted measures are calculated through a JavaScript-object being initialized upon start of the simulation. Though this process is not necessarily simulation-dependent, having all information ready and extracted for the purpose of simulating it makes access and reorganization of said information easier. The performance summary is not persisted to storage but rather displayed in the browser console window for simplicity reasons, but a proper storing of the summary to a json-file would certainly be a future enhancement. 


\subsection{Performance Results}

Beginning the project implementation, our hypothesis was that global car-agents - having access to information regarding all exact current edge-weights - would perform superior to local car-agents as they rely on solely local intrinsic information that does not capture the global network state. We did however expect that even local car-agents perform comparably well (discrepancies of not-too-high double-digit percentages), depending on the number of spawned travelling agents and intensity of network events.  

The primary test-simulation revolves around strictly local and strictly global car-agents.
A total of 50 car-agents are spawned (18 of type global, 32 of type local) and a number of network events occur regularly every few (approx. 2) seconds. \\
The overall average travelling time discrepancy between expected and actual travelling time was 0.62748 seconds with an average travelling time of 5.43866 seconds.
In comparison, the average travelling time discrepancy for local car-agents was 0.86022 seconds, while for global car-agents average discrepancy was only 0.21372 seconds.
Local car-agents had discrepancies of 0.025 - 6.612 seconds, with 8 out of 32 local car-agents arriving more than a full seconds later than expected and only 2 more arriving more than 0.1 seconds late.
Global car-agents had a discrepancy of 0.017 - 2.649 seconds, with 1 out of 18 global car-agents arriving more than a full seconds later than expected and 1 more arriving more than 0.1 seconds late.
No rerouting occurred for the rerouting-capable global car-agents. 

With $1/9$ global car-agents arriving more than 100 ms late, global car-agent performance was said to be excellent. The single outlier global car-agent with high discrepancy in travelling time was to heavy changes in the network and unfortunate positioning of the agent; this is however a very possible situation in real-world traffic and as such still a nice finding in the simulation. \\
With $1/4$ local car-agents arriving more than 100 ms late, performance of local car-agents is certainly worse than in case of global car-agents. They still (subjectively) traverse the network "well-enough", as they frequently arrive close or precisely on-time as well. \\
No possible rerouting of global car-agents is an indication of mostly-consistent network state, which was an intentional decision as more complex influences to the network would not allow a quite feasible visualization. For performance analysis, this statistic is not necessarily mandatory and the initial hypothesis is still inspected and validated nicely.
