\section{Related Work}\label{sec:relatedWork}

Simulating development of traffic is a well-traversed research topic regarding scheduling and network traversal simulation problems.
Throughout application, dedicated simulators have been applied frequently and early-on like the popular "Multi-agent Transport Simulator" (MATSim) \cite{sezen2003modeling} or "Repast Simphony" \cite{zargayouna2013agent} alongside dedicated extensions like the Symphony-based "SM4T Simnulator" \cite{ksontini2016building}.
Such applications provide advantages like automated timetables for public transport, advanced types of travelling agent and unified logging formats for simulation runs.
Resorting to holistic solutions like the above mentioned is especially useful for non-technological research regarding behavioral analysis or city planning \cite{brakewood2018literature}. If the multi-agent system aspects are predominant though (as is the case in this work), adapting similar implementation structures whilst doing the actual agent implementation work (done here) proves beneficial.

In contrast to fully incorporated applications, some approaches in literature already shift focus towards surrounding aspects surrounding of traffic simulation, like focusing on the distributing simulation \cite{mastio2015towards}.

As of \cite{mastio2015towards}, which depicts the simplest yet most fundamental approach to general network traversal, it utilizes a basic graph structure with vertices (also called "nodes") as intersection points and edges (also called "links") connecting these intersections. Agents are assigned a path over a fixed set of vertices as the shortest path over weighted edges. Path updates may occur only upon arrival at a given vertex when the agent then queries for a possible updated path.
The travel time $t$ on an edge depends on the number of agents currently on the edge in question. Calculating this $t$ can be done based on different kinds of functions modelling e.g. a certain free-flow capacity where for a given $n$ number of vehicles the weight ($t$) of an edge will not be impacted and only after reaching a certain threshold number $n_{\text{thresh}}$ the weight will increase to depict a slower movement (to a degree of $\alpha$) of traffic along said edge; exemplary formulas can, for example, be found in \cite{mastio2015towards} or \cite{ksontini2016building}.

To this agent-centric travel choice procedure, a literature review as is being depicted in the transport review of \cite{brakewood2018literature} adds a theoretical framework of traveler agent's perspective concerning travel- and mode choice as well as choices regarding route, boarding and departure. Incorporating this thinking, one ends up with a tight path-choosing procedure across multiple channels which theoretically boils down to the simple procedure of the previously mentioned formulas adapted to modes, routes and fixed schedules.
As this framework is a theoretical methodology at heart, the practical implementation aspect of this formal procedure raises multiple performance and architectural issues which without utilizing afore-mentioned well-established holistic framework solutions is expected to introduce bottlenecks.

As proposed by \cite{zargayouna2013agent}, agents adhere to a specific simulation workflow where they continuously query for updates regarding path choice whenever reaching nodes of the underlying network. Opposing the precise setup depicted here and utilizing in addition the thinking of \cite{mastio2015towards}, no precise time-step-based approach of procedurally checking all agents for their position but rather an event-based truly (distributed) multi-agent approach is implemented where agents query after complete time lapses instead of single short global time intervals. 

The visualization procedure of a given simulation then needs to adhere to the time lapse nature of the event logs, which not necessarily adheres well to standards defined by holistic framework solutions (like used in \cite{zargayouna2013agent}\cite{ksontini2016building}). As visualization is a side-aspect of most MAS-focused implementations (see \cite{mastio2015towards}), this is generally speaking seen as an added bonus for both debugging and reporting purposes.

Regarding performance metrics of a traffic simulation, metrics like use-of-transit, satisfaction and travelling-time are frequently proposed (\cite{brakewood2018literature}). With the use of dedicated car- and planner-agents (alongside \cite{zargayouna2013agent}) and the omittance of public transport time-table-based scheduling / availability-based carsharing, this leaves travelling-time as well as deviation from planning to execution performance as major indicators for performance.
In addition to this, subjective monitoring of network behavior may enhance the result (\cite{brakewood2018literature}, \cite{zargayouna2013agent}).
