\section{Related Work}\label{sec:relatedWork}

Simulating development of traffic is a well-traversed research topic regarding scheduling and network traversal simulation problems.
Throughout application, dedicated simulators have been applied frequently and early-on like the popular "Multi-agent Transport Simulator" (MATSim) \cite{sezen2003modeling} or "Repast Simphony" \cite{zargayouna2013agent} alongside dedicated extensions like the Symphony-based "SM4T Simnulator" \cite{ksontini2016building}.
Such applications provide advantages like automated timetables for public transport, advanced types of travelling agent and unified logging formats for simulation runs.
Resorting to holistic solutions like the above mentioned is especially useful for non-technological research regarding behavioral analysis or city planning \cite{brakewood2018literature}. If the multi-agent system aspects are predominant though (as is the case in this work), adapting similar implementation structures whilst doing the actual agent implementation work (done here) proves beneficial.

In contrast to fully incorporated applications, some approaches in literature already shift focus towards surrounding aspects surrounding of traffic simulation, like focusing on the distributing simulation \cite{mastio2015towards}.

% TODO: agent types / common architecture in literature (cars, planner)
%   - \cite{mastio2015towards}, \cite{brakewood2018literature}
\lipsum[1]

% TODO: experiment setup as of sources
\lipsum[2]
\lipsum[3]
\lipsum[4]

% TODO: some more ...
\lipsum[5]
\lipsum[6]
\lipsum[7]
\lipsum[8]



% \cite{brakewood2018literature}
% \cite{ksontini2016building}
% \cite{mastio2015towards}
% \cite{zargayouna2013agent}
% + own found ones